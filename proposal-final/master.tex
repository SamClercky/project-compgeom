\documentclass[11pt,a4paper]{article}
\usepackage[utf8]{inputenc}
\usepackage[english]{babel}
\usepackage[hidelinks]{hyperref}
\usepackage{amsmath}
\usepackage{calc}
\usepackage{listings}
\usepackage{color}
\usepackage{float}
\usepackage[edges]{forest}
\usepackage{tikz}
\usepackage{algorithm} 
\usepackage{algpseudocode}
\usepackage{multirow}

\usepackage[style=apa,sorting=nty]{biblatex}
\addbibresource{citations.bib}

\hypersetup{
    colorlinks=false,
    linkcolor=blue,
    filecolor=magenta
    }

\definecolor{dkgreen}{rgb}{0,0.6,0}
\definecolor{gray}{rgb}{0.5,0.5,0.5}
\definecolor{mauve}{rgb}{0.58,0,0.82}

\lstset{frame=tb,
  language=java,
  aboveskip=3mm,
  belowskip=3mm,
  showstringspaces=false,
  columns=flexible,
  basicstyle={\small\ttfamily},
  numbers=none,
  numberstyle=\tiny\color{gray},
  keywordstyle=\color{blue},
  commentstyle=\color{dkgreen},
  stringstyle=\color{mauve},
  breaklines=true,
  breakatwhitespace=true,
  tabsize=3
}

% font
\usepackage[T1]{fontenc}

% math modules
\usepackage{amsmath}
%\usepackage{amsfonts}
%\usepackage{amssymb}
\usepackage{parskip}

% images
\usepackage{graphicx}

% link
\usepackage{hyperref}

% floating objects
%\usepackage{float}

% Cover background
\usepackage[top=2cm, bottom=2cm, outer=0cm, inner=0cm]{geometry}
\usepackage[pages=some]{background}
\backgroundsetup{
scale=1,
color=black,
opacity=0.05,
angle=10,
position={12cm,-22cm},
contents={%
%   \includegraphics[height=20cm,width=20cm,keepaspectratio]{static/ULB-blason.png}
  }%
}
\usepackage{fullpage}


\begin{document}
\BgThispage
\title{
    \includegraphics[scale=0.2]{static/ULB.jpg}\\
    \vspace{8mm}

    \textbf{INFO-F420}\\
    \textbf{- Computational Geometry -}\\
    \text{Convex decomposition of simple polygons}
}
\author{
    Andreas Declerck, Célian Glénaz, and Mevel Gilles\\
    \vspace{10mm}
}
\date{\today}

\maketitle

% Remove page number
\thispagestyle{empty}
\newpage


\clearpage
\pagenumbering{arabic}

\section{Proposal}

For our project, we will construct an application demonstrating several
algorithms for decomposing a simple polygon.
This application will demonstrate the triangulation algorithm seen in class,
the algorithm from Chazelle [\cite{chazelle_decomposing_1979}], a greedy
decomposition without Steiner points, an algorithm that decomposes a
polygon into slabs along a certain direction and an algorithm that is based on
KD-trees to decompose a polygon [\cite{agarwal_polygon_2002}]. All algorithm
will be compared in their effectiveness for the construction of Minkowski sums
[\cite{agarwal_polygon_2002}].

For this project we would like to implement it in
\emph{Processing}\footnote{\url{https://processing.org/}} as it will make it
easier to work with without having to worry about the \emph{JavaScript} engine
being slow.
Doing the project in Java allows us also to get up and running more quickly as
our knowledge is better in this language.

We chose \verb|Processing| to still be close to the exercises seen in class, and
provide a visual representation of the discussed problem.

\section{Work distribution}

Celian:
\begin{itemize}
    \item Algorithm to calculate a Minkowski sum from a convex decomposition [\cite{agarwal_polygon_2002}]
    \item Triangulation of simple polygon
    \item Greedy convex decomposition [\cite{agarwal_polygon_2002}]
\end{itemize}

Andreas:
\begin{itemize}
    \item GUI
    \item Slab decomposition [\cite{agarwal_polygon_2002}]
    \item KD decomposition [\cite{agarwal_polygon_2002}]
\end{itemize}

Gilles:
\begin{itemize}
    \item Polynomial time algorithm for decomposing a polygon in its convex
        parts [\cite{chazelle_decomposing_1979}]
\end{itemize}

\printbibliography

\end{document}
