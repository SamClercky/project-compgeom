\documentclass[]{article}
\usepackage{vub}

\usepackage{color}

\usepackage{tcolorbox}
\tcbuselibrary{breakable}
\usepackage{hyperref}
\usepackage{footnote}
\usepackage{pgfplots}
\usepackage{csquotes}
\usepackage[style=apa,sorting=nty]{biblatex}
\addbibresource{citations.bib}

% Note: `vub` and `bruface` packages are now interchangeable for articles/reports.
\title{Proposal: Project Computational Geometry}
\subtitle{Approximating geometry}
\author{Andreas Declerck and Célian Glénaz}
\faculty{Sciences and Bio-Engineering Sciences} % Note: without the word "Faculty"!
\date{2023-2024}

%% handy macros
\newcommand*{\D}{\ensuremath{\mathcal{D}}}
\newcommand*{\T}{\ensuremath{\mathcal{T}}}
\newcommand*{\C}{\ensuremath{\mathcal{C}}}
\newcommand*{\R}{\ensuremath{\mathbb{R}}}
\newcommand*{\Q}{\ensuremath{\mathcal{Q}}}
\newcommand*{\Z}{\ensuremath{\mathcal{Z}}}
\newcommand*{\N}{\ensuremath{\mathcal{N}}}
\renewcommand*{\L}{\ensuremath{\mathcal{L}}}
\renewcommand*{\O}{\ensuremath{\mathcal{O}}}
\renewcommand*{\o}{\ensuremath{o}}
\newcommand*{\implies}{\ensuremath{\Longrightarrow}}
\renewcommand*{\inf}{\ensuremath{\infty}}
\newcommand*{\inherit}{\ensuremath{:<}}

\newcommand*{\boxed}[1]
{ \framebox{ \parbox[c]{0.9\textwidth}{
            #1
} } }

\newenvironment{thoughts}[0]
{\begin{tcolorbox}[breakable, title=Thoughts, colframe=red!75!black]
    }
    { 
    \end{tcolorbox}
    }

\newcommand*{\citationneeded}{\colorbox{orange}{[CITATION NEEDED]}}
% …

\begin{document}
\maketitle

\section{Introduction}

Our subject is about approximating geometric shapes with a set of convex hulls.
The method we would like to study comes from the paper
    [\cite{wei_approximate_2022}] which is used in software like \verb|Box2D|
to accelerate collision detection.

\section{Proposed project}

The project will be to implement the algorithms presented in the paper
    [\cite{wei_approximate_2022}].
The first goal will be to transform a hardcoded 3D object into convex-hulls
using these techniques.
Then we will import 3D files with more complex geometry, like print-in-place
objects used in 3D printing to validate its correctness.

The application will be accompanied by some controls to influence the algorithm,
and inspect the final object from several sides.

At first, we will not try to implement the collision part, as making sure that the
approximating algorithm is correct seems to be more important.
Later when we are confident that the approximating algorithm is consistent enough
with what we expect when comparing with the source model, collision detection
queries might be added to the GUI.

For this project we would like to implement it in
\verb|Processing|\footnote{\url{https://processing.org/}} as it will make it
easier to work with without having to worry about the \verb|JavaScript| engine
being slow, and having easier access to accelerators like \verb|OpenGL| (full
version).
Doing the project in Java allows us also to get up and running more quickly as
our knowledge is better in this language.

We chose \verb|Processing| to still be close to the exercises seen in class, and
being able to use Java at the same time.


\printbibliography

\end{document}
